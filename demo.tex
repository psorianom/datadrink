\documentclass[10pt]{beamer}

\usetheme{metropolis}
\usepackage{appendixnumberbeamer}

\usepackage{booktabs}
%\usepackage[scale=2]{ccicons}

\usepackage{soul}
\usepackage{color}
\usepackage{booktabs}
\usepackage{pgfplots}
\usepgfplotslibrary{dateplot}

\usepackage{xspace}
\newcommand{\themename}{\textbf{\textsc{metropolis}}\xspace}

\title{Pseudonymisation de documents textuels}
\subtitle{Le cas des décisions de justice}
\date{\textbf{Etalab Data Drink \#4} \\ \today}
\author{Pavel SORIANO-MORALES}
\institute{DINSIC -- ETALAB}
\titlegraphic{\hfill\includegraphics[height=1.5cm]{img/Etalab-logo-vectorise.pdf}}
\usetikzlibrary{patterns}
\begin{document}

\maketitle


\begin{frame}{Qu'est-ce que la pseudonymisation~?}
  	\metroset{block=fill}
  	\onslide<2->{
  	\begin{block}{Exemple JURITEXT000025735516}
  	\makebox[\textwidth]{\textbf{RÉPUBLIQUE FRANÇAISE}} 
	\makebox[\textwidth]{\textbf{AU NOM DU PEUPLE FRANÇAIS}}
  	\newline
	
	\textbf{COUR D'APPEL DE BASSE-TERRE}
	
	
	CHAMBRE SOCIALE \\
	ARRET No 153 DU SEIZE AVRIL DEUX MILLE DOUZE
	\newline
	
	\textbf{APPELANTE}
	
	Madame Rosine DU-NOM-FAUX\\
	123, rue de la Vieille-Lanterne\\
	97130 CAPESTERRE BELLE EAU \\
	née le 01 janvier 2000
	
	Représentée par Me Christiane ROMIL (TOQUE 119) [...]
	\end{block}
}


\end{frame}

\begin{frame}{Qu'est-ce que la pseudonymisation~?}
\metroset{block=fill}

\begin{block}{Exemple JURITEXT000025735516}
	\makebox[\textwidth]{\textbf{RÉPUBLIQUE FRANÇAISE}} 
	\makebox[\textwidth]{\textbf{AU NOM DU PEUPLE FRANÇAIS}}
	\newline
	
	\textbf{COUR D'APPEL DE BASSE-TERRE}
	
	
	CHAMBRE SOCIALE \\
	ARRET No 153 DU SEIZE AVRIL DEUX MILLE DOUZE
	\newline
	
	\textbf{APPELANTE}
	
	Madame \hl{Rosine DU-NOM-FAUX}\\
	\textcolor{red}{123, rue de la Vieille-Lanterne}\\
	
	\textcolor{red}{97130 CAPESTERRE BELLE EAU}\\
	\textcolor{green}{née le 01 janvier 2000}
	
	Représentée par Me \textcolor{orange}{Christiane ROMIL} (TOQUE 119) [...]
\end{block}

\end{frame}


\begin{frame}{Qu'est-ce que la pseudonymisation~?}
\metroset{block=fill}
\begin{block}{Exemple JURITEXT000025735516}
	\makebox[\textwidth]{\textbf{RÉPUBLIQUE FRANÇAISE}} 
	\makebox[\textwidth]{\textbf{AU NOM DU PEUPLE FRANÇAIS}}
	\newline
	
	\textbf{COUR D'APPEL DE BASSE-TERRE}
	
	
	CHAMBRE SOCIALE \\
	ARRET No 153 DU SEIZE AVRIL DEUX MILLE DOUZE
	\newline
	
	\textbf{APPELANTE}
	
	Madame \textcolor{blue}{B... D...}\\
	\textcolor{red}{...}\\
	\textcolor{green}{...}
	
	Représentée par Me \textcolor{orange}{E... F...} (TOQUE 119) [...]
\end{block}
\end{frame}



\begin{frame}{Problématique}
\begin{itemize}	
	\item<1-> Les décisions doivent être mises à disposition du public dans le respect de la vie privée {\textit{(Loi pour une République numérique)}}
	\item<2-> Près de 3,9 millions de décisions de justice par an
	\item<3-> La relecture reste très coûteuse 
	\item<4-> Besoin d'une solution automatisée
\end{itemize}
\end{frame}

\begin{frame}{Comment on fait~?}
	\begin{onlyenv}<1>\vspace*{\fill}\includegraphics[width=1\linewidth]{"img/nlp_diag"}\vspace*{\fill}\end{onlyenv}		\begin{onlyenv}<2>\vspace*{\fill}\includegraphics[width=1\linewidth]{"img/nlp_diag2"}\vspace*{\fill}\end{onlyenv}
	\begin{onlyenv}<3>\vspace*{\fill}\includegraphics[width=1\linewidth]{"img/nlp_diag3"}\vspace*{\fill}\end{onlyenv}
\end{frame}

\begin{frame}{Comment on fait~?}
\begin{onlyenv}<1>\vspace*{\fill}\includegraphics[width=1\linewidth]{"img/flow_ds"}\vspace*{\fill}\end{onlyenv}		\begin{onlyenv}<2>\vspace*{\fill}\includegraphics[width=1\linewidth]{"img/flow_ds2"}\vspace*{\fill}\end{onlyenv}
\end{frame}


\begin{frame}{Comment on fait~?}
\vspace{2cm}
\begin{onlyenv}<1>\vspace*{\fill}\includegraphics[width=1\linewidth]{"img/bilstm"}\vspace*{\fill}\end{onlyenv}
\begin{onlyenv}<2>\vspace*{\fill}\includegraphics[width=1\linewidth]{"img/bilstm1"}\vspace*{\fill}\end{onlyenv}
\begin{onlyenv}<3>\vspace*{\fill}\includegraphics[width=1\linewidth]{"img/bilstm2"}\vspace*{\fill}\end{onlyenv}
\begin{onlyenv}<4>\vspace*{\fill}\includegraphics[width=1\linewidth]{"img/bilstm3"}\vspace*{\fill}\end{onlyenv}
	


\end{frame}

\begin{frame}{Détails du corpus}
	\begin{center}

	\begin{tabular}{@{}lrrr@{}}
	\toprule
	\textbf{Dataset} & \textbf{\# de décisions} & \textbf{\# de phrases} & \textbf{Tokens} \\ \midrule
	train        & 57                           &  6 989                         & 173 448             \\
	dev          & 17                            & 2 257                         & 60 293        \\
	test         & 20                            & 1 963                         & 42 964        \\ \bottomrule
	\textbf{Totals}           & \textbf{94}                           & \textbf{11 209}                         & \textbf{276 705}\\ \bottomrule
\end{tabular}
	
	\begin{tabular}{@{}lrrrr|l@{}}
	\toprule
	\textbf{Dataset}&\textbf{AUX} & \textbf{DATE} & \textbf{LOC} & \textbf{PER} & \textbf{ALL} \\ \midrule
	train & 1 838         & 157           & 1 780         & 2 987         & 6 762         \\
	dev & 562          & 54            & 566          & 848          & 2 033         \\
	test & 545          & 26            & 434          & 629          & 1 634         \\ \bottomrule
	\textbf{Totals} &	\textbf{2 935}         & \textbf{237}            & \textbf{2 780}          & \textbf{4 464}          & \textbf{10 429}         \\ \bottomrule

\end{tabular}

\end{center}

\textbf{AUX:} avocats, membres de la formation de jugement \\
\textbf{DATE:} dates de naissance \textbf{LOC:} adresses de résidence\\
\textbf{PER:} parties et témoins.
\end{frame}


\begin{frame}{Résultats préliminaires}
%\begin{table}[]
%	\centering  
%	\begin{tabular}{@{}llllll@{}}
%		\toprule
%		& \multicolumn{5}{c}{\textbf{F-score}} \\ \midrule
%		\textbf{Model}         & \textbf{ALL} & \textbf{AUX} & \textbf{DATE} & \textbf{LOC} & \textbf{PER} \\ \midrule
%		\textbf{NoPreEmb}      & 94.46        & 95.53        & 100.0         & 91.50        & 94.67        \\
%		\textbf{Generic}   & 92.66        & 96.26        & 100.0         & 96.07        & 89.87        \\
%		\textbf{Domain}    & 96.60        & 96.96        & 100.0         & 97.84        & 96.05        \\
%		\textbf{elmo\_Domain}    & 95.95        & 94.59        & 100.0         & 98.28        & 95.92        \\ \bottomrule
%	\end{tabular}
%	\caption{F-score results for the configurations tested. Using the annotated corpus (train, dev, test). In bold the best values for each column.}
%	\label{tab:results}
%\end{table}
  \begin{figure}
	\begin{tikzpicture}
	\begin{axis}[
	mbarplot,
	xlabel={\textbf{Classe}},
	ylabel={\textbf{F-score}},
	width=0.9\textwidth,
	height=7cm,
	xtick={1,2,...,5},
	xticklabels={\textbf{ALL},AUX,DATE,LOC,PER},
	legend style={at={(1.3,1.1)},anchor=north east}]
	]
	% 1 = ALL, 2=AUX, 3=DATE, 4=LOC, 5=PER
	
		\addplot[pattern=dots] plot coordinates {(1, 94.46) (2, 95.53) (3, 100.0) (4, 91.50) (5, 94.67)};	
	\addplot[pattern=vertical lines] plot coordinates {(1, 92.66) (2, 96.26) (3, 100.0) (4, 96.07) (5, 89.87)};
	\addplot[] plot coordinates {(1, 96.60) (2, 96.96) (3, 100.0) (4, 97.84) (5, 96.05)};
	\addplot[pattern=horizontal lines] plot coordinates {(1, 95.95) (2, 94.59) (3, 100.0) (4, 98.28) (5, 95.92)};

	\legend{No PreEmbeddings, Generic PreEmbeddings, \textbf{Domain PreEmbeddings}, Elmo + Domain PreEmbeddings}
	
	\end{axis}
	\end{tikzpicture}
\end{figure}

\end{frame}	




\begin{frame}{PoC~: \texttt{pseudo.etalab.studio}}
\includegraphics[width=1\linewidth]{"img/pseudo"}
\end{frame}

\begin{frame}{What now~?}
\begin{columns}[T,onlytextwidth]
	\column{0.48\textwidth}
\begin{alertblock}{\textbf{Améliorer le modèle/système}}
	\begin{itemize}

		\item {Post-traitement à base de règles} %(utiliser le premier tag pour le même nom, ...)
		\item {Feature Engineering} %(utiliser des lexiques de noms/prénoms, syntaxe, ...)
		\item Corriger les erreurs orthographiques
		\item {Tester des autres algos} %(ElMo, BERT, GPT-2, ...)
	\end{itemize}
\end{alertblock}	
	\column{0.48\textwidth}
\begin{alertblock}{\textbf{Obtenir plus de données}}
	\begin{itemize}
		\item Génération de données synthétiques (fake data)
		\item Profiter des décisions déjà pseudonymisées
		\item Possibilité d'entraîner et tester sur un corpus beaucoup plus large
	\end{itemize}
\end{alertblock}
\end{columns}
\end{frame}

\begin{frame}{Générer fake data}
\begin{onlyenv}<1>
	\begin{citation}{}
		Le contrat de travail de Mme \textbf{X...} née le \textbf{[...]} demeurant \textbf{xxxxxxx}, passant à temps partiel
	sur une base de 20 heures 
	\end{citation}
\end{onlyenv}
\begin{onlyenv}<2>
	\begin{citation}{}
		Le contrat de travail de Mme \textbf{PER} née le \textbf{DATE} demeurant \textbf{LOC}, passant à temps partiel
		sur une base de 20 heures 
	\end{citation}
\end{onlyenv}

\begin{onlyenv}<3>
	\begin{citation}{}
		Le contrat de travail de Mme \textbf{DUPONT} née le \textbf{01 janvier 2018}, demeurant \textbf{99 rue Raoul Servant 69007 LYON}, passant à temps partiel
		sur une base de 20 heures 
	\end{citation}
\end{onlyenv}

\end{frame}
\begin{frame}[standout]

 Merci~! \\ 
 Des questions~?\\
 
 \vspace{1cm}
 
 \url{twitter.com/psorianom} 
 \url{github.com/psorianom} 
\end{frame}

\appendix

\begin{frame}{Analyse d'erreurs : AUX et LOC}
\begin{tabular}{@{}lll|lll@{}}
	\toprule
	\multicolumn{3}{c}{\textbf{Classe AUX}} &  \multicolumn{3}{c}{\textbf{Classe LOC}} \\
	\textbf{Token}    & \textbf{Real} & \textbf{Predicted} & \textbf{Token}    & \textbf{Real} & \textbf{Predicted} \\ \midrule
	signé             & O                 & O                      & \textbf{Unité}                & \textbf{B-LOC}             & \textbf{O}                  \\
	parMadame         & O                 & O                      & \textbf{Sud}            & \textbf{I-LOC}             & \textbf{O}                  \\
	\textbf{CALOT$^*$}    & \textbf{I-AUX}    & \textbf{O}             & \textbf{Secteur}       & \textbf{I-LOC}    & \textbf{O}             \\
	Conseiller        & O                 & O                      & \textbf{2}    & \textbf{I-LOC}    & \textbf{O}             \\
	en                & O                 & O                      & 	               &                   &                        \\
	l'                & O                 & O                      & Centre                & B-LOC                 & B-LOC                      \\
	absence           & O                 & O                      & Hospitalier             & I-LOC                 & I-LOC                      \\
	de                & O                 & O                      &                   &                   &                        \\ \bottomrule
\end{tabular}
\end{frame}

\begin{frame}{Analyse d'erreurs : PER}
\begin{center}


\begin{tabular}{@{}lll@{}}
	\toprule
	\multicolumn{3}{c}{\textbf{Classe PER}}\\
	\textbf{Token}    & \textbf{Real} & \textbf{Predicted}  \\ \midrule
	M.             & O                 & O                      \\
	Julien         & B-PER                 & B-PER                      \\
	Chavane    & I-PER    & I-PER         \\
	\textbf{de}        & \textbf{I-PER}                 & O                      \\
	\textbf{Roissy}                & \textbf{I-PER}                 & O                      \\ \bottomrule
\end{tabular}
\end{center}
\end{frame}

\begin{frame}[allowframebreaks]{References}

  \bibliography{demo}
  \bibliographystyle{abbrv}

\end{frame}

\end{document}
